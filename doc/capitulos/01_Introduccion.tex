\chapter{Introducción}

\section{Descripción y contexto}

{\color{blue}

El Internet de las Cosas, o IoT, es un sistema de dispositivos informáticos, máquinas mecánicas y digitales, objetos, animales o personas interrelacionados que cuentan con identificadores únicos (UID) y la capacidad de transferir datos a través de una red sin que sea necesaria la interacción entre personas o entre ordenadores. \cite{serpanos2018internet} \\

Una ``cosa`` en el Internet de las Cosas puede ser una persona con un implante de monitor cardíaco, un animal de granja con un chip, un automóvil que tiene sensores incorporados para alertar al conductor cuando la presión de los neumáticos es baja o cualquier otro objeto natural o artificial al que se le pueda asignar una dirección de Protocolo de Internet (IP) y que sea capaz de transferir datos a través de una red. Cada vez más, las organizaciones de diversos sectores utilizan el IoT para operar de forma más eficiente, comprender mejor a los clientes para ofrecerles un mejor servicio, mejorar la toma de decisiones y aumentar el valor del negocio. \\


El concepto de \textit{Internet of Things} fue propuesto en 1999 por el laboratorio de identificación automática del Instituto Tecnológico de Massachusetts (MIT). La UIT lo dio a conocer en 2005, empezando por China. El IoT puede definirse como \textit{``datos y dispositivos continuamente disponibles a través de Internet``}. La interconexión de ``cosas`` (objetos) que pueden dirigirse de forma inequívoca y las redes heterogéneas constituyen el IoT. La identificación por radiofrecuencia (RFID), los sensores, las tecnologías inteligentes y las nanotecnologías son los principales contribuyentes a al IoT para una variedad de servicios. Con la drástica reducción del coste de sensores y con la evolución de tecnologías como el ancho de banda, el procesamiento, los teléfonos inteligentes, la migración hacia el IPv6 y el 5G se está facilitando la adopción del IoT.\\

El IoT también ve todo como lo mismo, sin discriminar entre humanos y máquinas. Las ``cosas`` incluyen a los usuarios finales, los centros de datos (DC), las unidades de procesamiento, los teléfonos inteligentes, las tabletas, el Bluetooth, el ZigBee, la Asociación de Datos por Infrarrojos (IrDA), la banda ultraancha (UWB), las redes celulares, las redes Wi-Fi, los DC de comunicación de campo cercano (NFC), la RFID y sus etiquetas, los sensores y los chips, los equipos domésticos, los relojes de pulsera, los vehículos y las puertas de las casas. \cite{lea2020iot} \\

\section{Motivación}

Las personas de todo el mundo estan ya preparadas para disfrutar de las ventajas del Internet de las cosas (IoT). El IoT lo incorpora todo, desde el sensor corporal hasta la computación en la nube. Comprende los principales tipos de redes, como la distribuida, la de red, la ubicua y la vehicular, que han conquistado el mundo de la informática durante una década. Desde de estacionamiento de vehículos a su seguimiento, de la introducción de datos de pacientes a la observación de los pacientes a la observación de los pacientes, de la atención a los niños a la atención a los ancianos, de las tarjetas inteligentes a las tarjetas de campo cercano, los sensores están haciendo sentir su presencia. Los sensores desempeñan un papel fundamental en el IoT.\\

El IoT funciona en redes y estándares heterogéneos. Excepcionalmente, ninguna red está libre de amenazas y vulnerabilidades de seguridad. Cada una de las capas del IoT está expuesta a diferentes tipos de amenazas. Este proyecto se centra en las posibles amenazas que hay que abordar y mitigar para conseguir una comunicación segura en el IoT. \cite{hanes2017iot} \\

En otras palabras, el IoT combina ``lo real y lo virtual`` en cualquier lugar y en cualquier momento, atrayendo la atención tanto de desarrolladores como de ciberdelicuentes. Inevitablemente, dejar los dispositivos sin intervención humana durante un largo periodo podría dar lugar a robos. La seguridad ha sido finalmente reconocida como un requisito esencial para todo tipo de sistemas informáticos, incluidos los de IoT. Sin embargo, muchos sistemas IoT son mucho menos seguros que los típicos sistemas Windows/Mac/Linux. \\

Los problemas de seguridad de IoT se derivan de seguridad de la IO provienen de una serie de causas: características de seguridad inadecuadas en el hardware, software mal diseñado con una serie de vulnerabilidades, contraseñas por defecto y otros errores de de seguridad. Los nodos IoT inseguros crean problemas para la seguridad de todo el sistema IoT. Dado que los nodos suelen tener una vida útil de varios años, la gran base instalada de de dispositivos inseguros creará problemas de seguridad durante algún tiempo. Los nodos IoT inseguros son ideales para los ataques de denegación de servicio. El ataque Dyn es un ejemplo de ataque basado en el IoT contra la infraestructura tradicional de Internet. Los sistemas IoT inseguros también causan problemas de seguridad al resto de Internet. \\

La privacidad está relacionada con la seguridad, pero requiere medidas específicas a nivel de aplicación la red y los dispositivos. No sólo hay que proteger los datos de los usuarios contra el robo, sino que la red debe diseñarse de forma que los datos menos privados no puedan utilizarse fácilmente para inferir datos más privados. \cite{serpanos2018internet} \\ 

\section{Objetivos}

El principal objetivo de este proyecto es hacer uso de un entorno de desarrollo específico para llevar a cabo una implementación de un dispositivo IoT y analizar los potenciales problemas de seguridad que pueden aparecer durante la etapa de implementación.

\subsection{Objetivos específicos}

\begin{itemize}
    \item \textbf{OE.1:} Estudiar los distintos entornos de desarrollo para IoT y analizar funcionalidades y propiedades que se ajusten al proyecto.
    \item \textbf{OE.2:} Desarrollar un prototipo de aplicación para IoT utilizando un framework de desarrollo.
    \item \textbf{OE.3:} Explotación de una vulnerabilidad a nivel de dispositivo, protocolo, SO, aplicación o HW.
\end{itemize}

}