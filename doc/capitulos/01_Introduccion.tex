\chapter{Introducción}

Las personas de todo el mundo estan ya preparadas para disfrutar de las ventajas del Internet de las cosas (IoT). El IoT lo incorpora todo, desde el sensor corporal hasta la computación en la nube. Comprende los principales tipos de redes, como la distribuida, la de red, la ubicua y la vehicular, que han conquistado el mundo de la informática durante una década. Desde de estacionamiento de vehículos a su seguimiento, de la introducción de datos de pacientes a la observación de los pacientes a la observación de los pacientes, de la atención a los niños a la atención a los ancianos, de las tarjetas inteligentes a las tarjetas de campo cercano, los sensores están haciendo sentir su presencia. Los sensores también desempeñan un papel fundamental en el IoT.\\

El IoT funciona en redes y estándares heterogéneos. Excepcionalmente, ninguna red está libre de amenazas y vulnerabilidades de seguridad. Cada una de las capas del IoT está expuesta a diferentes tipos de amenazas. Este proyecto se centra en las posibles amenazas que hay que abordar y mitigar para conseguir una comunicación segura en el IoT.\\


El concepto de \textit{Internet of Things} fue propuesto en 1999 por el laboratorio de identificación automática del Instituto Tecnológico de Massachusetts (MIT). La UIT lo dio a conocer en 2005, empezando por China. El IoT puede definirse como \textit{"datos y dispositivos continuamente disponibles a través de Internet"}. La interconexión de cosas (objetos) que pueden dirigirse de forma inequívoca y las redes heterogéneas constituyen el IoT. La identificación por radiofrecuencia (RFID), los sensores, las tecnologías inteligentes y las nanotecnologías son los principales contribuyentes a al IoT para una variedad de servicios.\\

Con la drástica reducción del coste de las cosas, los sensores, el ancho de banda, el procesamiento, los teléfonos inteligentes y la migración hacia el IPv6, el 5G podría facilitar la adopción del IoT más de lo esperado.\\

El IoT también ve todo como lo mismo, sin discriminar entre humanos y máquinas. Las cosas incluyen a los usuarios finales, los centros de datos (DC), las unidades de procesamiento, los teléfonos inteligentes, las tabletas, el Bluetooth, el ZigBee, la Asociación de Datos por Infrarrojos (IrDA), la banda ultraancha (UWB), las redes celulares, las redes Wi-Fi, los DC de comunicación de campo cercano (NFC), la RFID y sus etiquetas, los sensores y los chips, los equipos domésticos, los relojes de pulsera, los vehículos y las puertas de las casas... En otras palabras, el IoT combina lo "fáctico y lo virtual" en cualquier lugar y en cualquier momento, atrayendo la atención tanto del "maker como del hacker". Inevitablemente, dejar los dispositivos sin intervención humana durante un largo periodo podría dar lugar a robos. IoT incorpora muchas cosas de este tipo. La protección era un problema importante cuando solo se acoplaban dos dispositivos. La protección para el IoT sería inimaginablemente compleja.\\

