\chapter{Estado del arte}

{\color{blue}

Antes de empezar a desarrollar los objetivos del proyecto es necesario conocer primero los fundamentos del Internet de las Cosas, esto engloba desde su funcionamiento, tipos de arquitecturas hasta los distintos estándares que existen. % Añadir parte de seguridad (normativas)

\section{Internet de las Cosas}

El Internet de las Cosas %(Ngu, Gutierrez, Metsis, Nepal, y Sheng, 2016)
permite que las personas
interactúen con dispositivos de distinto tipo conectados a Internet, como sensores o actuadores. Con
esto, se consigue integrar cualquier objeto del mundo real a Internet, estableciendo relaciones entre
todos ellos y cambiando la forma en la que interaccionamos con nuestro entorno. \\

La premisa básica y el objetivo de IoT es "conectar lo que no está conectado". Esto significa que los objetos que no están actualmente unidos a una red informática, es decir, a Internet, se conectarán para que puedan comunicarse e interactuar con personas y otros objetos. IoT es una transición tecnológica en la que los dispositivos nos permitirán sentir y controlar el mundo físico haciendo que los objetos sean más inteligentes y conectándolos a través de una red inteligente. 1 Cuando los objetos y las máquinas pueden ser detectados y controlados a distancia a través de una red, se consigue una mayor integración entre el mundo físico y los ordenadores. Esto permite mejoras en las áreas de eficiencia, precisión, automatización y habilitación de aplicaciones avanzadas. \\

El mundo del IoT es amplio y puede resultar algo complicado al principio debido a la abundancia de componentes y protocolos que engloba. En lugar de considerar IoT como un único tecnología, es bueno verlo como un paraguas de varios conceptos, protocolos y tecnologías, todos ellos dependientes a veces de un sector concreto. Aunque la amplia gama de elementos de IoT está diseñado para crear numerosos beneficios en las áreas de productividad y automatización, al mismo al mismo tiempo, introduce nuevos retos, como la ampliación del gran número de dispositivos y de la cantidad de datos que deben procesarse. que hay que procesar. \cite{hanes2017iot}

}