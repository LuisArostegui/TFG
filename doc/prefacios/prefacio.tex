\thispagestyle{empty}

\begin{center}
{\large\bfseries TFG - DASIoT: Desarrollo y Auditoría de Seguridad para prototipo de dispositivos IoT}\\
\end{center}
\begin{center}
Luis Aróstegui Ruiz\\
\end{center}

%\vspace{0.7cm}
\vspace{0.5cm}
\noindent{\textbf{Palabras clave}: Internet de las Cosas, Seguridad}\\

\vspace{0.7cm}
\noindent{\textbf{Resumen}}\\

En el contexto de IoT, hay un ecosistema de entornos de desarrollo específicos. En este TFG se hará uso de alguno de ellos para llevar a cabo la implementación de un dispositivo IoT y analizar los potenciales problemas de seguridad que pueden aparecer durante la etapa de implementación.

% El Internet de las Cosas (IoT) se utiliza para proporcionar conectividad entre diferentes
% dispositivos, los cuales están conectados a Internet. Es un sistema donde los objetos
% actúan con otros objetos a través de un medio de comunicación inalámbrico para
% intercambiar y transferir información sin interacción humana.
% Los dispositivos IoT son propensos a ataques vulnerables debido a la naturaleza simple
% y abierta de sus redes. Por lo tanto, la privacidad y la seguridad son la mayor
% preocupación de esta tecnología. El enfoque de las amenazas de seguridad y privacidad
% en IoT es crucial para promover el desarrollo de IoT.
% Este trabajo final de grado tiene como objetivo conceptualizar el Internet de las Cosas,
% establecer cuáles son sus principales características, y entrar en detalle en los aspectos
% de la privacidad y la seguridad de los datos personales. Cuáles son esas amenazas, la
% problemática particular del Internet de las Cosas que las genera y como mitigar el riesgo
% a dichas amenazas.
% En este trabajo nos haremos las siguientes preguntas: ¿Qué pasa con nuestros datos
% personales? ¿A qué riesgos estamos expuestos aportando tanta información “privada”?
% ¿Qué seguridad existe en el Internet de las Cosas? En este proyecto se pretende
% responder a estas y a muchas otras preguntas. Para ello vamos a profundizar en el tema
% de la privacidad y seguridad en el Internet de las Cosas.

\cleardoublepage


\begin{center}
{\large\bfseries Project Title: Project Subtitle}\\
\end{center}
\begin{center}
First name, Family name (student)\\
\end{center}

%\vspace{0.7cm}
\noindent{\textbf{Keywords}: Keyword1, Keyword2, Keyword3, ....}\\

\vspace{0.7cm}
\noindent{\textbf{Abstract}}\\

Write here the abstract in English.

\cleardoublepage
\thispagestyle{empty}

\noindent\rule[-1ex]{\textwidth}{2pt}\\[4.5ex]

Yo, \textbf{Nombre Apellido1 Apellido2}, alumno de la titulación TITULACIÓN de la \textbf{Escuela Técnica Superior
de Ingenierías Informática y de Telecomunicación de la Universidad de Granada}, con DNI XXXXXXXXX, autorizo la
ubicación de la siguiente copia de mi Trabajo Fin de Grado en la biblioteca del centro para que pueda ser
consultada por las personas que lo deseen.

\vspace{6cm}

\noindent Fdo: Nombre Apellido1 Apellido2

\vspace{2cm}

\begin{flushright}
Granada a X de mes de 201 .
\end{flushright}

\cleardoublepage
\thispagestyle{empty}

\noindent\rule[-1ex]{\textwidth}{2pt}\\[4.5ex]

D. \textbf{Nombre Apellido1 Apellido2 (tutor1)}, Profesor del Área de XXXX del Departamento YYYY de la Universidad de Granada.

\vspace{0.5cm}

D. \textbf{Nombre Apellido1 Apellido2 (tutor2)}, Profesor del Área de XXXX del Departamento YYYY de la Universidad de Granada.


\vspace{0.5cm}

\textbf{Informan:}

\vspace{0.5cm}

Que el presente trabajo, titulado \textit{\textbf{Título del proyecto, Subtítulo del proyecto}},
ha sido realizado bajo su supervisión por \textbf{Nombre Apellido1 Apellido2 (alumno)}, y autorizamos la defensa de dicho trabajo ante el tribunal
que corresponda.

\vspace{0.5cm}

Y para que conste, expiden y firman el presente informe en Granada a X de mes de 201 .

\vspace{1cm}

\textbf{Los directores:}

\vspace{5cm}

\noindent \textbf{Nombre Apellido1 Apellido2 (tutor1) \ \ \ \ \ Nombre Apellido1 Apellido2 (tutor2)}

\chapter*{Agradecimientos}
\thispagestyle{empty}

       \vspace{1cm}


Poner aquí agradecimientos...

