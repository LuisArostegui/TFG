\thispagestyle{empty}

\begin{center}
{\large\bfseries TFG - DASIoT: Desarrollo y Auditoría de Seguridad para prototipo de dispositivos IoT}\\
\end{center}
\begin{center}
Luis Aróstegui Ruiz\\
\end{center}

%\vspace{0.7cm}
\vspace{0.5cm}
\noindent{\textbf{Palabras clave}: Internet de las Cosas, Seguridad, MQTT, Framework, Privacidad, Raspberry PI}\\

\vspace{0.7cm}
\noindent{\textbf{Resumen}}\\

En el contexto de IoT, hay un ecosistema de entornos de desarrollo específicos. En este TFG se hará uso de alguno de ellos para llevar a cabo la implementación de un dispositivo IoT y analizar los potenciales problemas de seguridad que pueden aparecer durante la etapa de implementación. \\

Se analizarán los distintos frameworks para desarrollar una aplicación IoT usando una Raspberry PI, la cual enviará datos a la aplicación. Tras el desarrollo, se analizará la seguridad en el envió de datos entre el dispositivo y aplicación. \\

Debido al gran crecimiento del IoT, se busca conceptualizar el IoT, mostrando sus principales características y debilidades respecto a la privacidad y seguridad de datos para promover el desarrollo del IoT.

\cleardoublepage


\begin{center}
{\large\bfseries TFG - DASIoT: Development and Security Audit for IoT device prototyping}\\
\end{center}
\begin{center}
Luis Aróstegui Ruiz\\
\end{center}

%\vspace{0.7cm}
\noindent{\textbf{Keywords}: Internet of Things, Security, MQTT, Framework, Privacy, Raspberry PI, Security, Raspberry PI}\\

\vspace{0.7cm}
\noindent{\textbf{Abstract}}\\

In the context of IoT, there is an ecosystem of specific development environments. This final project will make use of some of them to carry out the implementation of an IoT device and analyse the potential security issues that may arise during the implementation stage. \\

Different frameworks will be analysed to develop an IoT application using a Raspberry PI, which will send data to the application. After the development, the security of the data sent between the device and the application will be analysed. \\

Due to the great growth of the IoT, the aim is to conceptualise the IoT, showing its main characteristics and weaknesses regarding privacy and data security in order to promote the development of the IoT.

\cleardoublepage
\thispagestyle{empty}

\noindent\rule[-1ex]{\textwidth}{2pt}\\[4.5ex]

Yo, \textbf{Luis Aróstegui Ruiz}, alumno de la titulación Ingeniería Informática de la \textbf{Escuela Técnica Superior
de Ingenierías Informática y de Telecomunicación de la Universidad de Granada}, con DNI 77944745E, autorizo la
ubicación de la siguiente copia de mi Trabajo Fin de Grado en la biblioteca del centro para que pueda ser
consultada por las personas que lo deseen.

\vspace{6cm}

\noindent Fdo: Luis Aróstegui Ruiz

\vspace{2cm}

\begin{flushright}
Granada a 04 de mes Julio de 2022 .
\end{flushright}

\cleardoublepage
\thispagestyle{empty}

\noindent\rule[-1ex]{\textwidth}{2pt}\\[4.5ex]

D. \textbf{Nombre Apellido1 Apellido2 (tutor1)}, Profesor del Área de XXXX del Departamento YYYY de la Universidad de Granada.

\vspace{0.5cm}

D. \textbf{Nombre Apellido1 Apellido2 (tutor2)}, Profesor del Área de XXXX del Departamento YYYY de la Universidad de Granada.


\vspace{0.5cm}

\textbf{Informan:}

\vspace{0.5cm}

Que el presente trabajo, titulado \textit{\textbf{Título del proyecto, Subtítulo del proyecto}},
ha sido realizado bajo su supervisión por \textbf{Nombre Apellido1 Apellido2 (alumno)}, y autorizamos la defensa de dicho trabajo ante el tribunal
que corresponda.

\vspace{0.5cm}

Y para que conste, expiden y firman el presente informe en Granada a X de mes de 201 .

\vspace{1cm}

\textbf{Los directores:}

\vspace{5cm}

\noindent \textbf{Nombre Apellido1 Apellido2 (tutor1) \ \ \ \ \ Nombre Apellido1 Apellido2 (tutor2)}

\chapter*{Agradecimientos}
\thispagestyle{empty}

       \vspace{1cm}
       
       A mis tutores, por todas las reuniones a las que se han ofrecido para ayudar con el desarrollo del proyecto. \\
       
       A mi madre por todo el esfuerzo que ha hecho para que yo pueda llegar a estudiar una carrera. \\
       
       A mi Beagle, por alegrarme cada día que llego a casa. \\
       
       A Nucleoo, por hacerme disfrutar tanto del trabajo y vida social. \\
       
       A mis colegas, por padrear tanto cada día. \\
      
